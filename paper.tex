\documentclass{article}

\usepackage{amsmath,amssymb,amsfonts,amsthm}

\newcommand{\A}{\mathcal{A}}
\newcommand{\F}{\mathbb{F}}
\newcommand{\Sq}{\mathrm{Sq}}

\newtheorem{prop}{Proposition}
\newtheorem{lem}{Lemma}

%TODO: frontmatter
\title{Determining the Structure of Length-$k$ Steenrod Operations as $A(r)$-Modules}
\author{Benjamin Kraft}

\begin{document}

\begin{abstract}
  The Steenrod Algebra $\mathcal{A}$ is the algebra of stable natural endomorphisms of the $\mathbb{Z}/2$-cohomology functor; it is generated by elements $Sq^{2^i}$.  Let $\mathcal{A}(k)$ be the subalgebra generated by the $Sq^{2^i}$ for $i\leq k$.  Consider the modules $L(k)$ spanned by sequences of Steenrod operations of length $k$.  Welcher proved that $L(k)$ is a free module over $\mathcal{A}(k-1)$.  We are interested in finding the structure of $L(k)$ as an $\mathcal{A}(r)$-module for any $r$.  We conjecture that $L(k)$ is built as an $\mathcal{A}(r)$-module out of $\mathcal{A}(r)/\!/\!\mathcal{A}(r-k)$, in the sense that it has an increasing filtration with quotients isomorphic to $\mathcal{A}(r)/\!/\!\mathcal{A}(r-k)$, and present partial results towards that claim.  In addition, we prove some interesting commutation relations in the Steenrod algebra relating to representations of Steenrod Algebra elements in Wood's $Z$-basis.
\end{abstract}

\section{The Commutation Relation}

  Recall that, in the Wood Z-basis, the top class of $A(n)\otimes_{A(n-1)}\F_2$ may be written as $\Sq^{2^{n+1}-2^n}\Sq^{2^{n+1}-2^{n-1}}\cdots\Sq^{2^{n+1}-1}$; denote this by $X_n$.

  \begin{lem}\label{lem:commutation-relation}
    For any positive integers $m$ and $n$,
    \[X_nX_{n-1}\Sq^{2^{n+1}m} = X_n\Sq^{2^{n+1}m}X_{n-1}.\]
  \end{lem}

  \begin{proof}
    First, we show the following identities by application of the Adem relations:
    \begin{prop}\label{prop:identity-1}
      For any positive integers $m$ and $n$,
      \[\Sq^{2^n-1}\Sq^{2^{n+1}m}=\Sq^{2^{n+1}m+2^n-1}.\]
    \end{prop}
    \begin{prop}\label{prop:identity-2}
      For any positive integers $k$, $m$, and $n$, with $k\leq n$,
      \[\Sq^{2^n-2^k}\Sq^{2^n-2^{k-1}+2^{n+1}m}=\Sq^{2^n-2^k+2^{n+1}m}\Sq^{2^n-2^{k-1}}+\Sq^{2^{n+1}-2^k}\Sq^{2^{n+1}m-2^{k-1}}.\]
    \end{prop}
    \begin{proof}[Proof of Proposition~\ref{prop:identity-1}]
      The Adem relations give us that
      \[\Sq^{2^n-1}\Sq^{2^{n+1}m} = \sum_{j=0}^{2^{n-1}-1}\binom{2^{n+1}m-1-j}{2^n-1-2j}\Sq^{2^{n+1}m+2^n-1-j}\Sq^j.\]
      Recall that in $\F_2$, $\binom{p}{q}$ is $1$ if and only if $p$ has a one in every position of its binary expansion that $q$ has a $1$.  Let $2^n>j>0$ and let $2^{\omega(j)}$ be the largest power of $2$ dividing $j$.  Then the binary expansion of $2^{n+1}m-1-j$ ends in a zero and then $\omega(j)$ ones, while that of $2^n-1-2j$ ends in a zero and then $\omega(j)+1$ ones, so the binomial coefficient in the Adem relation is zero.  Then the only term of the Adem relation that is nonzero is that when $j=0$, where $2^{n+1}m-1$ ends with $n+1$ ones, while $2^n-1$ consists of $n$ ones.
    \end{proof}
    \begin{proof}[Proof of Proposition~\ref{prop:identity-2}]
      We expand the first and last terms using the Adem relations, and show that they are identical except for the middle term. For convenience, let $d=2^{n+1}(m+1)-2^k-2^{k-1}$, the degree of the elements.
      \begin{align*}
        \Sq^{2^n-2^k}\Sq^{2^n-2^{k-1}+2^{n+1}m} &= \sum_{j=0}^{2^{n-1}-2^{k-1}} \binom{2^n-2^{k-1}+2^{n+1}m-1-j}{2^n-2^k-2j} \Sq^{d-j}\Sq^j \\
        \Sq^{2^{n+1}-2^k}\Sq^{2^{n+1}m-2^{k-1}} &=\sum_{j=0}^{2^n-2^{k-1}}\binom{2^{n+1}m-2^{k-1}-1-j}{2^{n+1}-2^k-2j}\Sq^{d-j}\Sq^j \\
      \end{align*}
      First, let $0\leq j\leq 2^{n-1}-2^{k-1}$; we will show the two binomial coefficients are equal.  We again consider the binary expansions.  The expansion of $2^n-2^k-2j$ is the same as that of $2^{n+1}-2^k-2j$, except that the leading one has been removed.  The corresponding bits of $2^n-2^{k-1}+2^{n+1}m-1-j$ and $2^{n+1}m-2^{k-1}-1-j$ are also zero and one, respectively, so the two binomial coefficients are the same.

      Now let $2^{n-1}-2^{k-1}<j<2^n-2^{k-1}$; we will show that the binomial coefficient
      \[\binom{2^{n+1}m-2^{k-1}-1-j}{2^{n+1}-2^k-2j}\]
      is zero.  Let $2^q$ be the next power of $2$ strictly larger than $2^n-2^{k-1}+j$, so $q\leq n-1$.  Then the $q$th position of the binary expansion will not satisfy the condition to make the binomial coefficient one, so it will be zero.

      Finally, if $j=2^n-2^{k-1}$, the binomial coefficient in the second formula has a zero on the bottom, thus is one, so we get the middle term of the claimed identity.
    \end{proof}

    Then in particular, by repeated application of the above identities, we may write
    \begin{align*}
      X_{n-1}\Sq^{2^{n+1}m} &= \Sq^{2^n-2^{n-1}}\Sq^{2^n-2^{n-2}}\cdots\Sq^{2^n-2}\Sq^{2^{n+1}m+2^n-1} \\
        &= \Sq^{2^n-2^{n-1}}\cdots\Sq^{2^n-4}\Sq^{2^{n+1}m+2^n-2}\Sq^{2^n-1} \\
        &\quad + \Sq^{2^n-2^{n-1}}\cdots\Sq^{2^n-4}\Sq^{2^{n+1}-2}\Sq^{2^{n+1}m-1}\\
        &\ \;\vdots \\
        &=\Sq^{2^{n+1}m}X_{n-1}\\
        &\quad + \Sq^{2^n-2^{n-1}}\cdots\Sq^{2^n-4}\Sq^{2^{n+1}-2}\Sq^{2^{n+1}m-1}\\
        &\quad + \cdots + \Sq^{2^n}\Sq^{2^{n+1}m-2^{n-1}}\Sq^{2^n-2^{n-2}}\cdots\Sq^{2^n-1}.
    \end{align*}
    Now all of the extra terms are products of (from left to right) an element of $\A(n-1)$, then an element of the form $\Sq^{2^{n+1}-2^k}$ for $k<n+1$, then some element of $\A$.  In particular, they are zero in $\A(n-1)\otimes_{\A(n)}\A$.  Then since $X_n$ is the top class of $\A(n)\otimes_{\A(n-1)}\F_2$, thus annihilates any element of $\A(n)$ that is not in $\A(n-1)$, $1\otimes x\mapsto X_nx$ is well-defined as a map from $\A(n-1)\otimes_{\A(n)}\A$ to $\A$, so the two sides are the same when multiplied by $X_n$:
    \[X_nX_{n-1}\Sq^{2^{n+1}m} = X_n\Sq^{2^{n+1}m}X_{n-1}.\]

  \end{proof}

\section{Representatives for \boldmath$\A(n)//\A(n-k)$}

  For convenience let $\bar{L}(k)=\bigcup_{i\leq k}L(i)$.

  \begin{prop}
    % There is a set of representatives for a basis of $\A(n)//\A(n-k)$ (in the sense of the above note), all of which lie in $\bar{L}(k)$.
    Any element $t$ of $\A(n)$ may be written as $x + yz$ where $x\in \bar{L}(k)$, $y\in \A$, and $z\in\A(n-k)\setminus \F_2$ for any $k\geq 0$.  Equivalently, $\A(n)\subset \bar{L}(k) + \A(\Sq^i)_{i\leq k}$, where we take $\A(\Sq^i)_{i\leq k}$ to be the left ideal of $\A$ generated by $\{\Sq^i\}_{i\leq k}$, for any $k\geq 0$.
  \end{prop}

  \begin{proof}
    It suffices to prove the proposition on some basis for $\A(n)$.  Recall that in Wood's Z-basis of the Steenrod algebra, a basis for $\A(n)$ consists of all strings of squares selected from~\eqref{eq:woodz} and multiplied in the given order:
    \begin{equation} \label{eq:woodz}
      \Sq^{2^{n+1}-2^n}\Sq^{2^{n+1}-2^{n-1}}\cdots\Sq^{2^{n+1}-1} \Sq^{2^n-2^{n-1}}\cdots\Sq^{2^n-1}\cdots\Sq^2\Sq^3\Sq^1
    \end{equation}

    We induct on the number of squares selected.  The empty product is $1$, which is in $\bar{L}(k)$, so we are done.  So now suppose the proposition holds on all elements of the Wood Z-basis of length at most $l-1$, and let $t$ be an element of length $l$.  Then, for some $m=2^j-2^i$, $i<j\leq n+1$, we may write $t=\Sq^m\left(x+yz\right)$ where $x$, $y$, and $z$ are as in the statement of the proposition.

    Now $\Sq^myz$ is clearly of the prescribed form, since $\Sq^my$ is in $\A$.  If $x\in\bar{L}(k-1)$, then $\Sq^m x\in \bar{L}(k)$ (since application of the Adem relations cannot increase the length of a monomial), so we may suppose $x\in L(k)$.

    Write $x$ in admissible form as $\Sq^{i_1}\Sq^{i_2}\cdots\Sq^{i_k}$.  If $\Sq^m\Sq^{i_1}\Sq^{i_2}\cdots\Sq^{i_k}$ is already in admissible form, then we have that $2^ki_k\leq 2^{k-1}i_{k-1}\leq\cdots\leq2i_1\leq m<2^{n+1}$, so in particular, $i_k<2^{n-k+1}$, so $i_k\in\A(\Sq^i)_{i\leq k}$, so in fact the entire product $\Sq^mx$ is in $\A(\Sq^i)_{i\leq k}$.  Then we may apply the Adem relations at least once, to get
    \[\Sq^m\Sq^{i_1}\Sq^{i_2}\cdots\Sq^{i_k} = a_0\Sq^{m+i_1}\Sq^{i_2}\cdots\Sq^{i_k} + \sum_{1\leq j\leq \frac{m}2}a_j\Sq^{m+i_1-j}\Sq^j\Sq^{i_2}\cdots\Sq^{i_k};\]
    the first term we know is in $L(k)$.  For each of the remaining terms, if we may repeat the same process, ignoring the first factor; either the product is in $\A(\Sq^i)_{i\leq k}$, or it may be further reduced by the Adem relations.  Finally, if we apply the Adem relations $k$ times, the rightmost factor must be $\Sq^i$ for $i\leq \frac{m}{2^k}<2^{n-k+1}$, so similarly the product lies in the ideal $\A(\Sq^i)_{i\leq k}$.  Then $\Sq^mx$ may be written in the desired form, so we are done.
  \end{proof}

  \section{Linear Independence}

  \begin{lem}\label{lem:linearly-independent}
    Taken as elements of $L(k)_{-\infty}$ as a module over $\A(n)\otimes_{\A(n-k)}\F_2$, the set
    \[\{X_nX_{n-1}\cdots X_{n-k+1}\Sq^I\mid I=(2^{n+1}m_1,2^nm_2,\ldots,2^{n-k+2}m_k), m_1\geq m_2\geq\cdots\geq m_{n-k}\}\]
    is linearly independent.
  \end{lem}
  \begin{proof}
    Note that by the commutation relation we have shown, the element $s=X_nX_{n-1}\cdots X_{n-k+1}\Sq^I$ may be rewritten as
    \[X_n\Sq^{2^{n+1}m_1}X_{n-1}\Sq^{2^nm_2}\cdots X_{n-k+1}\Sq^{2^{n-k+2}m_k}.\]
    Let $M_{k,n}$ be the subspace of $\A$ which is spanned by elements of the form $yz$ where $y\in L(k)$ and $z\in \A(n)$ with the degree of $z$ at least $1$.  If $b\in \A$, we will write $b+M_{k,n}$ for the subset of $\A$ consisting of elements of the form $b+m$ for $m\in M_{k,n}$.

    \begin{prop}\label{prop:m-thing}
      The element
      \[X_n\Sq^{2^{n+1}m_n}X_{n-1}\Sq^{2^nm_{n-1}}\cdots X_{n-k+1}\Sq^{2^{n-k+2}m_{n-k+1}}\]
      is in
      \[\Sq^{2^{n+1}(m_n+n)+1}\Sq^{2^n(m_{n-1}+n-1)+1}\cdots\Sq^{2^{n-k+2}(m_{n-k+1}+n-k+1)+1} + M_{k,n-k}.\]
    \end{prop}
    \begin{proof}

    We proceed by induction on $k$ for fixed $n$.  If $k=1$, we know that $X_n\in M_{1,n-1}+\Sq^{n2^{n+1}+1}$, since the latter term is the only element of $L(1)$ in the right dimension.  Then since, as an $\A(n)$-module, $L(1)$ is periodic in degrees modulo $2^{n+1}$, $X_n\Sq^{2^{n+1}m}\in M_{1,n-1}+\Sq^{2^{n+1}(n+m)+1}$.

    Now suppose the proposition holds for some $k$; we will show it for $k+1$.  Let 
    \[t=X_n\Sq^{2^{n+1}m_n}X_{n-1}\Sq^{2^nm_{n-1}}\cdots X_{n-k+1}\Sq^{2^{n-k+2}m_{n-k+1}},\]
    let $t'=X_{n-k}\Sq^{2^{n-k+1}m_{n-k}}$, let
    \[b=\Sq^{2^{n+1}(m_n+n)+1}\Sq^{2^n(m_{n-1}+n-1)+1}\cdots\Sq^{2^{n-k+2}(m_{n-k+1}+n-k+1)+1},\]
    and let $b'=\Sq^{2^{n-k+1}(m_{n-k}+n-k)+1}$.  Then by inductive hypothesis we know that $t\in b+M_{k,n-k}$ and by the base case we know that $t'\in b'+M_{1,n-k-1}$; we wish to show that $tt'\in bb'+M_{k+1,n-k-1}$.  To do this, it suffices to show that $bM_{1,n-k-1}$ and $M_{k,n-k}t'$ are contained in $M_{k+1,n-k-1}$; for simplicity we will show that this is true of the spanning sets we have constructed.
    
    First, let $y\in L(1)$, $z\in \A(n-k-1)$ with $z$ not in degree $0$, so that $yz\in M_{1,n-k-1}$.  Then by looking at degrees, $by$ is already in admissible form, so it is in $L(k+1)$, so $byz\in M_{k+1,n-k-1}$ as desired.  Since elements of the form $yz$ span $M_{1,n-k-1}$, we get that $bM_{1,n-k-1}\subset M_{k+1,n-k-1}$.

    Second, let $y\in L(k)$, $z\in \A(n-k)$ with $z$ not in degree $0$, so that $yz\in M_{k,n-k}$.  If $z\not\in\A(n-k-1)$, then $zX_{n-k}$ must be zero in $\A(n-k)$.  Then we may assume $z\in\A(n-k-1)$; we will show that both $yzb'$ and $yzM_{1,n-k-1}$ are contained in $M_{k+1,n-k-1}$.  Now if we write $zb'$ in admissible form, multiplying by $y$ will give an element already in admissible form by looking at degrees, so we need only show that $zb'\in M_{1,n-k-1}$.  If we multiply out using the Adem relations, this is clearly true, modulo showing that $zb'$ contains no terms of length $1$.  This is easy to show; by the Milnor basis, $z$ cannot end in a Steenrod square with degree a multiple of $2^{n-k}$, but $b'$ is a Steenrod square in a degree which is $1$ modulo $2^{n-k}$, so on multiplying the two, the length $1$ term must vanish, as desired.

    %maybe we can merge the following with the previous argument somehow?
    Finally, then, we must show that $yzM_{1,n-k-1}\subset M_{k+1,n-k-1}$; let $y'\in L(1)$ and $z'\in\A(n-k-1)$, so we must show that $yzy'z'\in M_{k+1,n-k-1}$.  Again looking at the Adem relations, we may write $zy'$ in the form $y''z''$ where $y''\in L(1)$ and $z''\in\A(n-k-1)$.  Then $z''z'\in \A(n-k-1)$ and is not in degree zero, and $yy''\in L(k+1)$, so $yy''z''z'\in M_{k+1,n-k-1}$.

    Then the induction is complete; all terms of $tt'-bb'$ are in $M_{k+1,n-k-1}$ as desired.
    \end{proof}

    Then since the elements in question may be written in the given form, their shortest terms are unique and distinct, so the terms are linearly independent.
  \end{proof}

\end{document}
